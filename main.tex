\documentclass[11pt]{moderncv}
\usepackage[utf8]{inputenc} % kodowanie ISO Latin2 (Pod Win cp1250)
%\usepackage[MeX]{polski}
\usepackage[polish]{babel}
\usepackage[below]{placeins}
% moderncv styles
%\moderncvstyle{casual}
%\moderncvstyle[nocolor]{casual}
\moderncvstyle[nocolor]{classic}
\usepackage{xpatch}
\xpatchcmd\cventry{,}{}{}{}
%\moderncvstyle[roman]{classic}

% character encoding
%\usepackage[utf8]{inputenc}

% the following definition is from the file moderncvstyleclassic.sty


% personal data
\firstname{Mariusz }
\familyname{Wójcik}
%\title{Design enthusiast\dots}
%\address{12 somestreet\\3456 somecity} % for classic style
\address{ul. Cybisa 3 m.24\\02-784 Warszawa} % for casual style
\phone{504 00 31 31}
\email{mw.wojcik@gmail.com}
%\extrainfo{{\small maried with childrens}}
%\quote{Any intelligent fool can make things bigger, more complex, and more violent. It takes a touch of genius -- and a lot of courage -- to move in the opposite direction.}

%\renewcommand{\listsymbol}{{\fontencoding{U}\fontfamily{ding}\selectfont\tiny\symbol{'102}}}
\definecolor{see}{rgb}{0.5,0.5,0.5}

% command for text subscripts
\newcommand{\up}[1]{\ensuremath{^\textrm{\scriptsize#1}}}

% the ConTeXt symbol
\def\ConTeXt{%
  C%
  \kern-.0333emo%
  \kern-.0333emn%
  \kern-.0667em\TeX%
  \kern-.0333emt}

% slanted small caps (only with roman family; the sans serif font doesn't exists :-()
%\usepackage{slantsc}
%\DeclareFontFamily{T1}{myfont}{}
%\DeclareFontShape{T1}{myfont}{m}{scsl}{ <-> cork-lmssqbo8}{}
%\usefont{T1}{myfont}{m}{scsl}Testing the font


%----------------------------------------------------------------------------------
%            content
%----------------------------------------------------------------------------------
\begin{document}
\maketitle
%\makequote
\section{O mnie}
\cvitem{}{
Jestem programistą... z wyboru, pasji i zamiłowania. Uwielbiam się uczyć i ciągle pracuję nad poprawą jakości wytwarzanego kodu. Dużo czytam i szanuję rzetelną wiedzę. 
Jestem zdeklarowanym entuzjastą metodyk bazujących na konceptach Domain Driven Design i technikach Event Storming. Uwielbiam dyskusje o wzorcach projektowych i architekturze aplikacji.  
}


\closesection



\section{Doświadczenie}

\cvitem{2010-}{\textbf{Asseco Poland}}
\cvitem{Rola}{\emph{Architekt, Projektant JEE,Programista, Lider Obszaru Merytorycznego}}
\cvitem{Obowiązki}{\small 
\begin{itemize}
\item Brałem udział w dwóch projektach integracyjnych klasy enterprise realizujących koncepcję SOA (ESB) przy wykorzystaniu Java/JEE i WebService.  
\item Byłem współautorem projektu architektonicznego oraz implementacji aplikacji przeznaczonej do użycia w środowisku rozproszonym (modularny monolit, docelowo mikrousługi).
\item Realizowałem projekty z wykorzystaniem konceptów Domain Driven Design oraz technik Event Storming.
\item Współtworzyłem narzędzia wspierające proces wytwarzania aplikacji (analizatory i generatory kodu).
\item Byłem liderem zespołu deweloperskiego (5-8 osób). 
\end{itemize}}
\cvitem{Osiągnięcia}{
	\begin{itemize}
		\item  \small Brałem udział w projektowaniu i implementacji aplikacji dokonującej translacji kodu żródłowego(z Flex na Java) w oparciu o bibliteki do parsowania i analizy semantycznej kodu (ANTL+JavaParser). Narzędzie stało się podstawą skutecznej i w dużej mierze zautomatyzowanej migracji kodu pomiędzy wymienionymi platformami ( w skali ok. 1000 artefaktów wejściowych).  
		\item Współtworzyłem narzędzie dokonujące analiy kodu żródłowego projektu, a następnie ekstrakcji i wizualizacji  zależności pomiędzy artefaktami (Python+NetworkX+Graphviz). Materiał ten pozwalał na  optymalizację pracy zespołu (unikanie wzajemnych blokad). Aplikacja została użyta w praktyce i wsparła proces migracji kodu.
		\item Uczestniczyłem w rozwoju aplikacji realizującej automatyczne, spójne i powtarzalne przekształcanie artefaktów analityczynych zgromadzonych w repozytoriach Enterprise Architect na artefakty deweloperskie (JAVA/XSD/WSDL) (skala ok 1200 obiektów). Narzędzie było używane w procesie wytwarzania aplikacji. 
\end{itemize}}

\cvitem{2007-2010}{\textbf{ABG}}
\cvitem{Rola}{\emph{Projektant JEE,Programista JEE}}
\cvitem{Opis}{\small 
\begin{itemize}
\item Brałem udział w projekcie programistycznym wykorzystującym technologie:JEE (Spring,JPA-Hibernate,JSP,JSF) - PROW713. 
\end{itemize}}

\cvitem{2004-2007}{\textbf{Centralny Ośrodek Informatyki Politechniki Warszawskiej}}
\cvitem{Rola}{\emph{Projektant/Programista portalu i aplikacji internetowych}}
\cvitem{Opis}{\small 
\begin{itemize}
\item Brałem udział w projektach programistyczny wykorzystujących technologie:  Java (Servlet+JSP),Spring,Struts, Hibernate,JSTL,PL-SQL
\end{itemize}}
\closesection

\newpage 

\section{Projekty własne}
\cventry{}{}{\href{https://github.com/mwwojcik/airline-reservation-system}{Airline Reservation System}}{Kompleksowy przykład użycia technik DDD oraz Event Storming do modelowania dziedziny}{\newline(https://github.com/mwwojcik/airline-reservation-system)}{}
\newline
\cventry{}{}{
	\href{https://github.com/mwwojcik/ml_workspace_kotlin/blob/master/dokumentacja/raport-silnik-regul.pdf}{Realizacja prostego silnika reguł walidacji przy użyciu technik NLP}}{Aplikacja dokonująca przekształcenia prostych reguł walidacyjnych definiowanych w języku naturalnym na kod wykonywalny w języku Kotlin (Python+Kotlin)}{\newline \small
	(https://github.com/mwwojcik/ml\_workspace\_kotlin/blob/master/dokumentacja/raport-silnik-regul.pdf)}{}

\section{Przebieg nauki}
\cventry{1999-2005}{Politechnika Warszawska, studia dzienne magistersko-inżynierskie na \\Wydziale Elektrycznym, kierunek Informatyka;}{}{}{}{{}{}}
\cventry{1995-1999}{Liceum Ogólnokształcące im. Marii Skłodowskiej-Curie - klasa o profilu Informatyka w Puławach}{}{}{}{}{{}{}}
\closesection


\section{\textbf{Przebyte szkolenia i certyfikaty}}
\cventry{11.2020}{SmartTesting}{Sprytne testy jednostkowe}{}{}{}
\cventry{03.2020}{DNA}{Droga Nowoczesnego Architekta}{}{}{}
\cventry{08.2016}{SCJP}{Oracle Certified Professional, Java SE 8 Programmer }{}{}{}
\cventry{10.2008}{SCJP}{Sun Certified Java Programmer}{}{}{}
\cventry{07.2009}{SCWCD}{Sun Certified Web Component Developer}{}{}{}
\cventry{02.2010}{SCBCD}{Sun Certified Business Component Developer}{}{}{}
\closesection


\section{Języki obce}
\cvlistitem{Angielski - dobry w mowie i piśmie}
\closesection





\footnotesize
Wyrażam zgodę na przetwarzanie moich danych osobowych zawartych w mojej ofercie pracy dla potrzeb niezbędnych do realizacji procesu rekrutacji (zgodnie z Ustawą z dn. 29 sierpnia 1997 o Ochronie Danych Osobowych Dz. U. nr 133 pozycja 883).
\end{document}
