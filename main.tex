\documentclass[11pt]{moderncv}
\usepackage[utf8]{inputenc} % kodowanie ISO Latin2 (Pod Win cp1250)
%\usepackage[MeX]{polski}
\usepackage[polish]{babel}
\usepackage[below]{placeins}
% moderncv styles
%\moderncvstyle{casual}
%\moderncvstyle[nocolor]{casual}
\moderncvstyle[nocolor]{classic}
\usepackage{xpatch}
\xpatchcmd\cventry{,}{}{}{}
%\moderncvstyle[roman]{classic}

% character encoding
%\usepackage[utf8]{inputenc}

% the following definition is from the file moderncvstyleclassic.sty


% personal data
\firstname{Mariusz }
\familyname{Wójcik}
%\title{Design enthusiast\dots}
%\address{12 somestreet\\3456 somecity} % for classic style
\address{ul. Cybisa 3 m.24\\02-784 Warszawa} % for casual style
\phone{504 00 31 31}
\email{mw.wojcik@gmail.com}
%\extrainfo{{\small maried with childrens}}
%\quote{Any intelligent fool can make things bigger, more complex, and more violent. It takes a touch of genius -- and a lot of courage -- to move in the opposite direction.}

%\renewcommand{\listsymbol}{{\fontencoding{U}\fontfamily{ding}\selectfont\tiny\symbol{'102}}}
\definecolor{see}{rgb}{0.5,0.5,0.5}

% command for text subscripts
\newcommand{\up}[1]{\ensuremath{^\textrm{\scriptsize#1}}}

% the ConTeXt symbol
\def\ConTeXt{%
  C%
  \kern-.0333emo%
  \kern-.0333emn%
  \kern-.0667em\TeX%
  \kern-.0333emt}

% slanted small caps (only with roman family; the sans serif font doesn't exists :-()
%\usepackage{slantsc}
%\DeclareFontFamily{T1}{myfont}{}
%\DeclareFontShape{T1}{myfont}{m}{scsl}{ <-> cork-lmssqbo8}{}
%\usefont{T1}{myfont}{m}{scsl}Testing the font


%----------------------------------------------------------------------------------
%            content
%----------------------------------------------------------------------------------
\begin{document}
\maketitle
%\makequote

\section{Przebieg nauki}
\cventry{1999-2005}{Politechnika Warszawska, studia dzienne magistersko-inżynierskie na \\Wydziale Elektrycznym, kierunek Informatyka ukończone z wynikiem bardzo dobrym;}{}{}{}{{}{}}
\cventry{1995-1999}{Liceum Ogólnokształcące im. Marii Skłodowskiej-Curie - klasa autorska o profilu Informatyka w Puławach}{}{}{}{}{{}{}}

\section{Doświadczenie zawodowe}

\cvitem{2010-}{\textbf{Asseco Poland}}
\cvitem{Rola}{\emph{Projektant JEE,Programista JEE, Lider Obszaru Merytorycznego, Kierownik Zespołu}}
\cvitem{Opis}{\small 
\begin{itemize}
\item Udział w projektach integracyjnych tworzonych w technologii Java, JEE, SOA. 
\end{itemize}}

\cvitem{2007-2010}{\textbf{ABG}}
\cvitem{Rola}{\emph{Projektant JEE,Programista JEE}}
\cvitem{Opis}{\small 
\begin{itemize}
\item Udział w projekcie programistycznym wykorzystującym technologie:JEE (Spring,JPA-Hibernate,JSP,JSF) - PROW713  
\end{itemize}}

\cvitem{2004-2007}{\textbf{Centralny Ośrodek Informatyki Politechniki Warszawskiej}}
\cvitem{Rola}{\emph{Projektant/Programista portalu i aplikacji internetowych}}
\cvitem{Opis}{\small 
\begin{itemize}
\item Udział w projektach programistyczny wykorzystujących technologie:  Java (Servlet+JSP),Spring,Struts, Hibernate,JSTL,PL-SQL
\item Udział w projektach analitycznych z zakresu Inżynierii Oprogramowania opartych na języku modelowania UML
\end{itemize}}

\section{\textbf{Przebyte szkolenia i certyfikaty}}
\cventry{03.2017}{FrontEnd}{Szkolenia online w zakresie HTML5,CSS3,JavaScript,JQuery organizowane przez serwis www.eduweb.pl}{}{}{}
\cventry{08.2016}{SCJP}{Oracle Certified Professional, Java SE 8 Programmer }{}{}{}
\cventry{10.2008}{SCJP}{Sun Certified Java Programmer}{}{}{}
\cventry{07.2009}{SCWCD}{Sun Certified Web Component Developer}{}{}{}
\cventry{02.2010}{SCBCD}{Sun Certified Business Component Developer}{}{}{}


\closesection
\newpage 


\section{Projekty}

\cvitem{2015-    \newline 2010-2012}{\textbf{KRUS-WFL}}
\cvitem{Pracodawca}{Asseco Poland}
\cvitem{Technologie}{JAVA, SOA, Enterprise Architect API ,SOAP WebService}
\cvitem{Opis}{ Projekt integracyjny wykonywany według koncepcji SOA WebService, którego celem  było wytworzenie warstwy integracyjnej pomiędzy systemami realizującymi obsługę świadczeń rolniczych KRUS.Projekt wdrożony. Aplikacja używana przez 3000 użytkowników.}
\cvitem{Zadania}{
	\begin{itemize}
		\item Lider obszaru merytorycznego (4-10 osób)
		\item Opracowanie zasad wykorzystania Kanonicznego Modelu Dziedziny do definicji kontraktu pomiędzy integrowanymi systemami
		\item Uzgadnianie kontraktu pomiędzy systemami (projekt EA)
		\item Opracowanie narzędzi służących do utrzymania spójności modelu (skrypty EA)
		\item Opracowanie zasad, wdrożenie i utrzymywanie dwuetapowej generaty modelu (aplikacje java przekształcające model EA do WSDL/XSD, a następnie generujące kod Java, pluginy maven, konfiguracja pom)
		\item Implementacja	
	\end{itemize}
}


\cvitem{2012-2015}{\textbf{Elektroniczna Platforma Gromadzenia, Analizy i Udostępniania zasobów cyfrowych o Zdarzeniach Medycznych (P1)}}
\cvitem{Pracodawca}{Asseco Poland}
\cvitem{Technologie}{ JAVA, SOA, Enterprise Architect API ,SOAP WebService}
\cvitem{Opis}{ Projekt integracyjny wykonywany według koncepcji SOA WebService, którego celem  było wytworzenie kompleksowej platformy wspierającej proces leczenia pacjentów. }
\cvitem{Zadania}{
\begin{itemize}
	\item Szef mikrozespołu programistów (4-5 osób)
	\item Opracowanie zasad, wdrożenie i utrzymywanie dwuetapowej generaty modelu (aplikacje java przekształcające model EA do WSDL/XSD, a następnie generujące kod Java, pluginy maven, konfiguracja pom)
	\item Implementacja	obszaru merytorycznego - skierowania
	\item Prace implementacyjne przy wprowadzeniu międzynarodowego formatu zapisu zdarzeń medycznych HL7
\end{itemize}
}






\cvitem{2007-2010}{\textbf{PROW713}}
\cvitem{Pracodawca}{ABG}
\cvitem{Technologie}{ Oracle DB, JAVA, Spring, JSF, JPA (Hibernate)}
\cvitem{item}{Aplikacja służąca do dystrybucji funduszy unijnych w ramach Programu Restrukturyzacji Obszarów Wiejskich 2007-2013}
\cvitem{Zadania}
{
\begin{itemize}
	\item Programista Java
	\item Prace implementacjyjne we wszystkich warstwach aplikacji DB/DAO/Usług/JSF
\end{itemize}
}


\closesection

\newpage

\section{Podprojekty autorskie}

\cvitem{Asseco Poland}{\textbf{Genertor MDA (PIM2PIM)}}
\cvitem{Technologie}{Java,Sparx API do Enterprise Architect}
\cvitem{Opis}{ narzędzie służące do przetwarzania artefaktów projektowych zdefiniowanych w modelu Enterprise Architect na artefakty WSDL/XSD }

\cvitem{Asseco Poland}{\textbf{Skrypty Enterprise Architect}}
\cvitem{Technologie}{Sparx API do Enterprise Architect}
\cvitem{Opis}{ Skrypty pomagające utrzymanie spójności artefaktów projektowych modelu EA(repozytorium ok. 10 tys obiektów)}

\cvitem{Asseco Poland}{\textbf{Narzędzie do monitorowania postępów prac w projekcie}}
\cvitem{Technologie}{(Java, Java FX,Apache POI,JIRA Rest API}
\cvitem{Opis}{Narzędzie wykorzystywane do monitorowania postępu prac programistycznych na podstawie JIRA wykorzystywane przy metodzie pomiaru zaawansowania EVM}



%W ramach projektu uczesniczyłem przy opracowaniu zasad współpracy i uzgadnianiu kontraktu pomiędzy integrowanymi systemami. 

%Brałem aktywny udział przy tworzeniu narzędzi pomocniczych używanych w procesie wytwórczym (Generator MDA służący do tworzenia artefaktów WSDL/XSD na %podstawie modelu zdefiniowanego w EA), oraz narzędzia do monitorowania postępu prac programistycznych na podstawie JIRA}{}




\closesection




\section{Języki obce}
\cvlistitem{Angielski - dobry w mowie i piśmie}



\section{Umiejętności dodatkowe}
\cvcomputer{OS}{Linux, Windows}{Środowiska programistyczne}{InteliJ Idea}
\cvcomputer{Technologie}{JSP,JSF,Hibernate,JSTL,JPA,Spring,JEE}{}{}
\cvcomputer{WWW}{HTML5,CSS3,JQuery}{Bazy danych}{PostgreSQL,Oracle}
\cvcomputer{Języki projektowania}{UML}{Środowiska projektowe}{Enterprise Architect}
\cvcomputer{Metodyki projektowe}{SCRUM, EVM}{}{}


\closesection

%\section{Section with your own content}\closesection
%Your content here,\\
%inside the normal \LaTeX{} environment.

%\emptysection{}
%\cvitem{Now}{Back to moderncv layout, without making a new section :-)}
%\vspace{1,5cm}

\footnotesize
Wyrażam zgodę na przetwarzanie moich danych osobowych zawartych w mojej ofercie pracy dla potrzeb niezb?dnych do realizacji procesu rekrutacji (zgodnie z Ustaw? z dn. 29 sierpnia 1997 o Ochronie Danych Osobowych Dz. U. nr 133 pozycja 883).
\end{document}
